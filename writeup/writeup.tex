%%%%%%%%%%%%%%%%%%%%%%%%%%%%%%%%%%%%%%%%%%%%%%%%%%%%%%%%%%%%%%%%%%%%%
%
% CSCI 1430 Writeup Template
%
% This is a LaTeX document. LaTeX is a markup language for producing
% documents. Your task is to fill out this
% document, then to compile this into a PDF document.
%
% TO COMPILE:
% > pdflatex thisfile.tex
%
% For references to appear correctly instead of as '??', you must run
% pdflatex twice.
%
% If you do not have LaTeX and need a LaTeX distribution:
% - Departmental machines have one installed.
% - Personal laptops (all common OS): www.latex-project.org/get/
%
% If you need help with LaTeX, please come to office hours.
% Or, there is plenty of help online:
% https://en.wikibooks.org/wiki/LaTeX
%
% Good luck!
% James and the 1430 staff
%
%%%%%%%%%%%%%%%%%%%%%%%%%%%%%%%%%%%%%%%%%%%%%%%%%%%%%%%%%%%%%%%%%%%%%
%
% How to include two graphics on the same line:
%
% \includegraphics[\width=0.49\linewidth]{yourgraphic1.png}
% \includegraphics[\width=0.49\linewidth]{yourgraphic2.png}
%
% How to include equations:
%
% \begin{equation}
% y = mx+c
% \end{equation}
%
%%%%%%%%%%%%%%%%%%%%%%%%%%%%%%%%%%%%%%%%%%%%%%%%%%%%%%%%%%%%%%%%%%%%%%%%%%%%%%%%%%%%%%%%%%%%%%%%

\documentclass[11pt]{article}

\usepackage[english]{babel}
\usepackage[utf8]{inputenc}
\usepackage[colorlinks = true,
            linkcolor = blue,
            urlcolor  = blue]{hyperref}
\usepackage[a4paper,margin=1.5in]{geometry}
\usepackage{stackengine,graphicx}
\usepackage{fancyhdr}
\setlength{\headheight}{15pt}
\usepackage{microtype}
\usepackage{times}
\usepackage{booktabs}

% python code format: https://github.com/olivierverdier/python-latex-highlighting
\usepackage{pythonhighlight}

\frenchspacing
\setlength{\parindent}{0cm} % Default is 15pt.
\setlength{\parskip}{0.3cm plus1mm minus1mm}

\pagestyle{fancy}
\fancyhf{}
\lhead{Homework 3 Writeup}
\rhead{CSCI 1430}
\rfoot{\thepage}

\date{}

\title{\vspace{-1cm}Homework 3 Writeup}


\begin{document}
\maketitle
\vspace{-2cm}
\thispagestyle{fancy}

\section*{Project Overview}

In this project, we look at scene geometry by learning how to estimate the camera projection matrix and the fundamental matrix. We then use RANSAC, after estimating the fundamental matrix, to find a fundamental matrix with the most inlier matches between two images. After we find the matches the the estimated projection matrices, we can then calculate the 3D points that the corresponding image coordinates.

\section*{Implementation Detail}
Task 1: \\
Our first task for this project was to recover a matrix that transforms 3D world coordinates to 2D image coordinates given 2D to 3D point correspondences. Our goal was to find M by setting up a system of linear equations by using linear least squares regression.\\

Task 2: \\
Next, we were tasked with finding points on our object across images so that we can discover their 3D position by finding 2D correspondences across views. Here we use RANSAC to estimate a fundamental matrix from a point calculated by a feature point detector. As described by the handout, RANSAC does the following:
\begin{enumerate}
  \item Randomly select a subset of point correspondences 
  \item Solve for the fundamental matrix using the subset
  \item Count the number of inliers of the calculated fundamental matrix
\end{enumerate} 

Task 3: \\
This task asks us to calculate the 3D points the matches correspond to. We had to use np.linalg.lstsq() to solve for the 3D points given information about two points and two projection matrices.\\

Task 4: \\
Lastly, in this task, we were asked to estimate the mapping of points in one image to lines in another via the fundamental matrix. As explained in the handout, we were able to solve a system of homogeneous linear equations by using Singular Value Decomposition and taking the solution of the fundamental matrix by taking the row to the smallest singular value. We then decompose F using singular value decomposition into the matrices to then estimate a rank 2 matrix. With this done, we are easily able to calculate the fundamental matrix.

\section*{Result}
Here are the final results for this project: \\
\\ Mikes and Ikes:
\begin{figure}[h]
    \centering
    \includegraphics[width=15cm]{images/mi1.png}
    \includegraphics[width=15cm]{images/mi2.png}
    \includegraphics[width=15cm]{images/mi3.png}
    \caption{This is the image that gets displayed when you run "python main.py --sequence mikeandikes" in the terminal.}
    \label{fig:result1}
\end{figure}

\newpage

Cards:
\begin{figure}[h]
    \centering
    \includegraphics[width=15cm]{images/cards1.png}
    \includegraphics[width=15cm]{images/cards2.png}
    \includegraphics[width=15cm]{images/cards3.png}
    \caption{These are the images that get displayed when you run "python main.py --sequence cards" in the terminal.}
    \label{fig:result1}
\end{figure}

\newpage

Dollar:
\begin{figure}[h]
    \centering
    \includegraphics[width=15cm]{images/dollar1.png}
    \includegraphics[width=15cm]{images/dollar2.png}
    \includegraphics[width=15cm]{images/dollar3.png}
    \caption{These are the images that get displayed when you run "python main.py --sequence dollar" in the terminal.}
    \label{fig:result1}
\end{figure}

\end{document}
